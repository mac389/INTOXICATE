\documentclass[10pt]{article}
\input{hicss-packages.tex}
\newcommand{\sansserifformat}[1]{\fontfamily{cmss}{ #1}}%

\setlength\titlebox{7cm}

% You can expand the titlebox if you need extra space
% to show all the authors. Please do not make the titlebox
% smaller than 5cm (the original size).
% 5cm is the default for a single row list of authors

% title for manuscript without author info 
\title{Understanding Complex Medical Decision Models: Equivalence Classes (PRELIM)}

% title for final manuscript
% comment for initial manuscript
% \title{Detailed Formatting Guidelines for Preparing a Final Manuscript with Author Information: Paper May Not Exceed Ten Pages Including References}

% Comment this for initial manuscript 
% Uncomment this for final manuscript
% \author{Row 1 Column 1 Author \\
%  Affiliation Name \\
%  {\underline{ email@domain}} \\ \\
%  Row 2 Column 1 Author \\
%  Affiliation Name \\
%  {\underline{ email@domain} } \\ \And
%  Row 1 Column 2 Author \\
%  Affiliation Name \\
%  {\underline{ email@domain} } \\ \\
%  Row 2 Column 2 Author\\
%  Affiliation Name \\
%  {\underline{ email@domain} } \\ \And
%  Row 1 Column 3 Author \\
%  Affiliation Name \\
%  {\underline{ email@domain} } \\ \\
%  Row 2 Column 3 Author\\
%  Affiliation Name \\
%  {\underline{ email@domain} } \\ }

\date{}

\begin{document}
\maketitle
\begin{abstract}
A deeper understanding of how models predict complex phenomenon provides a structure to understand their inner working and to communicate that inner rationale to stakeholders not directly involved in model building. Here we demonstrate how identifying equivalence classes can 
\end{abstract}

\subsubsection*{Keywords:}

Include up to five keywords that capture the main topics or themes of the paper. Separate each keyword with a comma and space.

\section{Introduction}\label{sec:intro}

\subsection{Clinical Decision Support Algorithms Are Intrinsically Complex}
\subsection{They Must Also Be Interpretable}
\subsection{Equivalence Classes Identify Functionally Equivalent Phenotypes}
\subsection{Alternative Approaches}
\subsection{Case Study\textemdash INTOXICATE: An Algorithm to Reduce Unnecessary Admissions to the Intensive Care Unit}
\begin{itemize}
  \item Partial Dependency Plot
  \item Unit Tests
\end{itemize}

\section{Methods}
\subsection{Implementation of INTOXICATE}
\subsection{Calculation of Equivalence Classes}
- determining of radius
\subsection{Communication of Results to Medical Practitioners}

\section{Results}

\section{Conclusions}

\section{Type-style and fonts}
\label{sec:type-style}

\begin{itemize}
\setlength\itemsep{0em}
\item Resolution: 600 dpi
\item Color Images: Bicubic Downsampling at 300dpi
\item Compression for Color Images: JPEG/Medium Quality
\item Grayscale Images: Bicubic Downsampling at 300dpi
\item Compression for Grayscale Images: JPEG/Medium Quality
\item Monochrome Images: Bicubic Downsampling at 600dpi
\item Compression for Monochrome Images: CCITT Group 4
\end{itemize}

If your paper contains many large images they will be down-sampled to reduce their size during the conversion process.  However the automated process used will not always produce the best image, and you are encouraged to perform this yourself on an image by image basis. The use of bitmapped images such as those produced when a photograph is scanned requires significant storage space and must be used with care.

% For one-column wide figures use
\begin{figure}[thb]
	% Use the relevant command to insert your figure file.
	% For example, with the graphicx package use
    \centering
    % \includegraphics[trim={3cm 3cm 3cm 3cm}, clip,width=0.9\linewidth]{sample-image}
	% figure caption is below the figure
	\caption{Sample figure with caption.}
	\label{fig: sample-figure}       % Give a unique label
\end{figure}

\subsection{Second-order headings}
 
As in this heading, they should be Times 11-point boldface, initially capitalized, flush left, with one blank line before, and one after. 

%\section{Fonts}

%\begin{table}[thb]
%\centering
%\caption{\label{font-table} Font guide. \vskip 3pt }
%\label{tab: fonts}
%\begin{tabular}{l|rl}
%\hline \bf Type of Text & \bf Font Size & \bf Style \\ \hline
%paper title & 14 pt &  \bf bold \\
%authors & 10 pt &  \underline{email} underlined \\
%abstract title & 12 pt &  \bf bold\\
%abstract text & 10 pt &  \it italic\\
%section titles & 12 pt & \bf bold \\
%subsection titles & 11 pt & \bf bold \\
%document text & 10 pt  & \\
%captions & 9 pt & \sansserifformat{\captionsize sans-serif, \bf bold} \\
%bibliography & 9 pt & \\
%footnotes & 8 pt & \\
%\hline
%\end{tabular}
%\end{table}


\section{References} 

References and in-text citation should be in line with the format recommended by the Publication Manual of the American Psychological Association (7th edition). The style and grammar guidelines are freely available and can be found at:  \url{https://apastyle.apa.org/style-grammar-guidelines}.

List and number all bibliographical references in 9-point Times, single-spaced, and in an alphabetical order at the end of your paper. For example, \textcite{Castells2010, Allen1997} and \textcite{Bloomberg2018} and \textcite{Allen1997}.

% if added before the last page, this command can help balancing columns
%\addtolength{\textheight}{-.2cm} 

% Bibliography 
\bibliographystyle{apalike}
\bibliography{sample}

\printbibliography

\end{document}
